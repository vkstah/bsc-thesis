\chapter{Johdanto} \label{johdanto}

%Tutkimuskysymys 1: Minkälaisia ongelmia tilanhallintaan liittyy?
%Tutkimuskysymys 2: Minkälaisia käytännölliseksi todettuja keinoja ja työkaluja tilanhallinnan helpottamiseksi on olemassa?

Yhden sivun web-sovellukset (engl. single-page application) ovat tuoneet mukanaan uudenlaisen tavan luoda käyttäjäystävällisiä ja käytännöllisiä verkkoympäristöjä ja sovelluksia. Perinteisissä monen sivun verkkosivustoissa ilmenneitä sekä suorituskykyyn että käyttökokemukseen vaikuttaneita ongelmia ja puutteita on onnistuttu ratkaisemaan yhden sivun web-sovelluksien avulla. Tyypillinen perinteisiin monen sivun verkkosivustoihin liittynyt ongelma on esimerkiksi tarpeettoman monet palvelinpyynnöt, jolloin jokaisella sivun latauskerralla käyttäjän selain lataa joko osittain tai kokonaan täysin samat resurssit palvelimelta. Yhden sivun web-sovelluksessa sivu käytännössä ladataan kerran, ja täytetään tarvittaessa uudelleen dynaamisesti. Tällöin käyttäjän ei tarvitse jokaisella latauskerralla noutaa resursseja uudestaan.

Tilanhallinta on yksi keskeisimpiä web-sovelluksen kehittämisessä vastaan tulevia kysymyksiä. Tilan on oltava saatavilla sitä tarvitsevissa komponenteissa, ja sitä on tietyissä tapauksissa päästävä muuttamaan sovelluksen eri osista käsin. Sulavan kehitysprosessin edistämiseksi kehittäjän tulee huomioida nämä asiat jo komponenttien suunnitteluvaiheessa. Sovelluksen kasvaessa suuremmaksi komponentit tyypillisesti etääntyvät toisistaan huomattavasti, jolloin alkuperäinen ratkaisu ei enää välttämättä ole käytännöllinen tai edes toimiva. Kokenut web-sovelluksiin perehtynyt kehittäjä tunnistaa ja osaa ennakoida tilanhallintaan liittyviä ongelmia sekä ratkaista niitä myös jälkikäteen.

Tässä tutkielmassa on määrä esittää lukijalle holistinen näkemys tilanhallinnasta. Tilanhallintaa tarkastellaan Metan kehittämän React JavaScript -kirjaston näkökulmasta. Tutkimuskysymykset ovat: (1) \textit{Minkälaisia ongelmia tilanhallintaan liittyy?} ja (2) \textit{Minkälaisia käytännölliseksi todettuja keinoja ja työkaluja tilanhallinnan helpottamiseksi on olemassa?}

Tutkielma on toteutettu kirjallisuuskatsauksena. Tutkittavaan aiheeseen liittyvää aineistoa on haettu informaatioteknologian kannalta olennaisista tietokannoista, kuten Web of Science, ACM sekä IEEE/IEE. Tutkielman luvussa \ref{reactjs} käsitellään React-kehityksen kannalta olennaisia käsitteitä. Luvussa \ref{tilanhallinta} avataan tilanhallinnan avainkäsitteet ja esitetään konkreettisia esimerkkejä tilanhallinnan perusoperaatioista. Tilanhallinnassa usein esiintyviä ongelmia käsitellään luvussa \ref{Tilanhallinnan ongelmat} esimerkkien kautta. Luvussa \ref{Tilanhallinnan menetelmät} käsitellään tilanhallinnan menetelmiä ja niiden käyttötapauksia. Lisäksi menetelmiä sovelletaan luvussa \ref{Tilanhallinnan ongelmat} käsiteltyihin ongelmiin. Tutkielman johtopäätökset esitetään lopuksi luvussa \ref{Johtopäätökset}.