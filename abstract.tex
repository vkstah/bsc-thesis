\keywords{React, React Hooks, web-sovellukset, tilanhallinta, Flux-arkkitehtuuri, Redux}
\begin{abstract}
Yhden sivun web-sovellukset ovat uudistaneet verkkokehittämistä uusilla innovatiivisilla ratkaisuilla. Perinteisiin monen sivun verkkosivustoihin verrattuna JavaScript-pohjaiset yhden sivun web-sovellukset mahdollistavat kustannustehokkaamman ja vuorovaikutteisemman kokemuksen käyttäjälle. Jotta tämä olisi mahdollista, tulee sovelluksessa olla yksi tai useampi tila, joka voi muuttua esimerkiksi vastauksena käyttäjän antamaan syötteeseen. Tässä tutkielmassa tarkastellaan web-sovelluksen tilaa sekä sen hallintaa Metan kehittämän React JavaScript -kirjaston näkökulmasta. Tutkielmassa käsitellään tilanhallintaan liittyviä perusoperaatioita, haasteita sekä hyödyllisiksi ja toimiviksi todettuja menetelmiä ja työkaluja. 

Tutkielma on toteutettu kirjallisuuskatsauksena. Tutkielmassa käsitellään tilanhallinnassa tyypillisesti esiintyviä ongelmia sekä tilanhallinnan menetelmiä. Esitettyihin ongelmiin pyritään tutkielman aikana esimerkkien johdattelemana esittämään ratkaisuja tilanhallinnan menetelmien avulla.

Tutkielman johtopäätöksenä voidaan todeta, että modernien web-sovellusten tilanhallintaan liittyviin ongelmiin on olemassa tilannekohtaisia ratkaisuja, joita tutkielmassa eritellään. Menetelmiä ja ratkaisuja ovat esimerkiksi tilanhallintaan suunnatut ulkoiset kirjastot sekä komponenttien kokoonpanon laatiminen tietyllä tavalla. Yhtä oikeaa ratkaisua ei ole. Sopivan menetelmän valinta ja käyttöönotto perustuu pitkälti sovelluksen monimutkaisuuteen ja laajuuteen. Menetelmän toimivuuden lisäksi sovelluksen kehittäjän pohdittavaksi jää sovelluksen kokoluokalle ja monimutkaisuudelle sopivan ratkaisun valinta.
\end{abstract}

% Ehkä viimeisessä kappaleessa voisi vielä jotenkin
% todeta -- vähän rautalangasta vääntämällä -- että
% tutkielman tuloksena myös eritellään sitä, mikä
% tilanhallinta ratkaisu sopii mihinkin tilanteeseen.
% Muuten on ehkä vaarana se, että annetaan tiivistelmän 
% lukijalle vähän lattea "ratkaisuja on monia erilaisia"
% -mielikuva ilman aiheen syvempää käsittelyä, mikä ei tietysti
% tee tutkielman viimeisille luvuille oikeutta
%
% -SR
